%!TEX root = csc236-a3.tex
\question{18}

For a point $x \in \Q$ and a closed interval 
$I = [a,b]$, $a, b \in \Q$, we say that 
$I$ {covers} $x$ if $a \le x \le b$. 
Given a set of points $S = \{x_1,\dots,x_n\}$ 
and a set of closed intervals $Y = \{I_1,\dots,I_k\}$ 
we say that $Y$ covers $S$ if every point 
$x_i$ in $S$ is covered by some interval $I_j$ in $Y$.

In the ``Interval Point Cover'' problem, we are 
given a set of points $S$ and a set of closed 
intervals $Y$. The goal is to produce a minimum-size 
subset $Y' \subseteq Y$ such that $Y'$ covers $S$.

Consider the following greedy strategy for the problem.

\begin{minipage}{\linewidth}
    \begin{algorithm}[H]
        \caption{$Cover(S,Y)$}
        \begin{algorithmic}[1]

            \vspace{.5 em}
            \Require 
            \Statex
                $S$ is a finite collection of points in $\Q$.
                $Y$ is finite set of closed intervals which covers $S$.
            \Ensure
            \Statex Return a subset $Z$ of $Y$ such that
            $Z$ is the smallest subset of $Y$ which covers $S$.
            \vspace{.5 em}

            \State $L = \{x_1,\dots,x_n\} \gets S$ sorted 
            in nondecreasing order
            \State $Z \gets \emptyset$
            \State $i \gets 0$
            \While{$i < n$}
            \If{$x_{i+1}$ is not covered by some interval in $Z$}
                \State $I \gets$ interval $[ a, b ]$ in $Y$ 
                which maximizes $b$ subject to $[a, b]$ containing $x_{i+1}$
                \State $Z$.append($I$)
                \State $i \gets i + 1$
            \EndIf
            \EndWhile

            \State\Return $Z$

            \vspace{.5 em}
        \end{algorithmic}
    \end{algorithm}
\end{minipage}

\vspace{1 em}
Give a complete proof of correctness for $Cover$
subject to its precondition and postcondition.

\vspace{1 em}

\textbf{Proof of correctness}

\begin{proof}
    \textbf{1. Define the loop invariant}
    
    We define our loop invariant Inv(Z, i):
    \begin{itemize}
        \item [(a)] $Z = \{ I_0, I_1, \dots, I_i \}$ is a subset of $Y$
        \item [(b)] $0 \leqslant i \leqslant len(L)$
        \item [(c)] $Z$ may be extended to $Z'$ such that
        \begin{itemize}
            \item [(\emph{i})] $Z'$ covers points in $L$ up to $x_i$
            \item [(\emph{ii})] $Z'$ is the smallest such subset that covers points in $L$
        \end{itemize}
    \end{itemize}
    
    \newpage
    
    \base 
    
    \textbf{2. Establish the loop invariant}
    
    Assume that the preconditions are true. Then when the loop is reached, we have
    \begin{itemize}
        \item $Z = \emptyset$
        \item $i = 0$
    \end{itemize}
    Thus conjunct (a) holds since $\emptyset \subseteq Y$, (even if $Y$ was empty).
    Conjunct (b) is obviously true. Also, conjunct (c) holds because $i = 0$, so
    $x_i \not \in L$ and so property (\emph{i}) says that $Z'$ covers points
    in $L$ up to $x_i$, but since there are no points in $L$ to cover yet, it is
    true that $Z'$ covers points in $L$ up to $x_i$ (which is actually nothing yet).
    Similarly, $Z' = \emptyset$ is the 
    smallest such subset that covers no points in $L$. Thus conjunct (c) (\emph{ii}) also
    holds.
    
    \istep
    
    \textbf{3. Maintain the Loop Invariant}
    
    We want to show that if the LI holds before an iterative step, 
    then LI holds at the end of the iterative step.
    
    Assume that the preconditions are satisfied and that the LI holds
    at the start of an arbitrary iteration. Let $Z_0$ and $i_0$ be the 
    values of $Z$ and $i$ at the start of an arbitrary loop iteration.
    Also assume that $i_0 < len(L)$ so that another loop is necessary. 
    Let $Z_1$ and $i_1$ be the values of $Z$ and $i$ at the end of that 
    loop iteration. We are assuming that $Inv(Z_0, i_0)$ is true and we
    want to show that $Inv(Z_1, i_1)$ is also true.
    
    By our assumption $Inv(Z_0, i_0)$, we know $Z_0 = \{ I_0, \dots, I_{i_{0}} \}$
    can be extended to $Z_0' = \{ I_0, \dots, I_{i_{0}}, I_{i_{1}} \}$
    that satisfies (\emph{i}) and (\emph{ii}).
    Further, from the body of the loop we know that
    \begin{itemize}
        \item $Z_1 = \{ I_0, \dots, I_{i_{0}}, I_{i_{0}+1} \}$, if $Z_0'$ is extended
        to $Z_1$
        \item $Z_1 = \{ I_0, I_1, \dots, I_{j} \}$, otherwise
        \item $i_1 = i_0 + 1$
    \end{itemize}
    holds. By the first two bullet points above, in either case, it follows 
    that $Z_1 \subseteq Y$, so conjunct (a) is true.
    By the third bullet point above, and by our assumption that
    $i_0 < len(L)$, it follows that $i_0 + 1 \leqslant len(L)
    \implies i_1 \leqslant len(L)$. Hence conjunct (b) is true.
    We now need to show conjunct (c):
    
    $Z_1$ may be extended to $Z_1'$ such that
    \begin{itemize}
        \item [(\emph{i})] $Z_1'$ covers points in $L$ up to $x_{i_{1}}$
        \item [(\emph{ii})] $Z_1'$ is the smallest such subset that covers points in $L$
    \end{itemize}
    If it's the case that $Z_0'$ extends $Z_1$, then $Z_1$ already contains the
    interval $I_{i_{0} + 1} = I_{i_{1}}$ being appended, so it follows that
    $Z_1$ covers all the points in $L$ up to $x_{i_{1}}$, and conjunct (c) is 
    satisfied.
    
    On the other hand, suppose $Z_0'$ doesn't extend $Z_1$ (bullet point two), then
    let $I_j$ denote an interval in $Z_0$ such that $\abs{I_{i_{1}}} < \abs{I_j}$
    and $I_j$ covers up to the left-most point $x_{i_{1}}$ in $L$.
    Our goal here is to show that the point $x_{i_{1}}$ is covered in $Z_1$.
    Thus we know that the interval $I_{i_{1}}$ is not appended because the point
    $x_{i_{1}}$ is already covered by $I_j$ which is contained in $Z_0$. Hence
    it follows that the interval $I_j$ is also contained in $Z_1$. Hence
    $Z_1$ covers all the points in $L$ up to $x_{i_{1}}$ as desired.
    Further, since $I_j$ is the interval that extends to the right as far
    as possible, it must extend at least as far to the right as $I_{i_{1}}$.
    Therefore $I_j$ covers as many points as $I_{i_{1}}$, then it follows that
    $Z_1$ covers all the same points as $Z_0'$ and since $Z_0'$ is an optimal 
    solution, it follows that $Z_1$ is optimal (the smallest such subset).
    
    \textbf{4. LI and exit condition $\implies$ postcondition}
    
    When the loop terminates, then $i = n$. Also suppose that 
    $Inv(Z, i)$ holds. Then $Z = \{ I_0, \dots, I_m \}$
    for $0 \leqslant m \leqslant n$ and $Z$ covers all the points 
    in $L$. Since $L$ contains the same points as $S$, and since 
    $Z$ is the smallest such subset of $Y$ that covers points in 
    $S$, then we can conclude that the post condition is satisfied.
    
    \textbf{5. Termination}
    
    Let the measure of progress be defined as $m(L, i) = len(L) - i$.
    Since $i$ increases by one after each iteration, 
    we can see that $len(L) - i$
    decreases by one each iteration. Further, from the invariant,
    we know that $i \leqslant len(L)$, so it follows that
    $len(L) - i \geqslant 0$. Then by our assumption $i = len(L)$, so 
    we have $len(L) - len(L) = 0$ and the loop terminates.
    
    
\end{proof}



\newpage