%!TEX root = csc236-a3.tex
\def \cL{\mathcal{L}}

\question{16}
In the following, for each language $L$ over the alphabet $\Sigma = \{0,1\}$ construct a 
regular expression $R$ and a DFA $M$ such that $\mathcal{L}(R) = \mathcal{L}(M) = L$. Prove 
the correctness of your DFA. 

	\subquestion{8}
	Let $L_1 = \{x \in \{0,1\}^*: \mbox{ the first and last characters of $x$ are the same}\}$. 
	Note: $\epsilon \notin L$ since $\epsilon$ does not have a first or last character.
	
	We define the regular expression $R = 0(0 + 1)^*0 + 1(0 + 1)^*1$ such that 
	$\cL(R) = L_1$. That is, $L_1$ is the regular language matched by the regular expression
	$R$.
	
	Next, we define the DFA $M$:
	
	\vspace{5mm}
	
	\begin{tikzpicture}[shorten >=1pt,node distance=3cm,on grid,auto] 
        \node[state,initial] (q_0)                {$q_0$};
        \node[state, accepting] (q_1) [above right=of q_0]  {$q_1$};
        \node[state] (q_3) [right=of q_1]                   {$q_3$};
        \node[state, accepting] (q_2) [below right=of q_0]  {$q_2$};
        \node[state] (q_4) [right=of q_2]                   {$q_4$};
        \path[->] 
        (q_0) edge [bend left] node {0} (q_1)
              edge [bend right] node [swap] {1} (q_2)
        (q_1) edge [bend left] node  {1} (q_3)
              edge [loop above] node {0} ()
        (q_2) edge [bend left] node {0} (q_4)
              edge [loop below] node {1} ()
        (q_3) edge [bend left] node {0} (q_1)
              edge [loop right] node {1} ()
        (q_4) edge [bend left] node {1} (q_2)
              edge [loop right] node {0} ();
    \end{tikzpicture}
    
    \textbf{Proof of Correctness}
    \begin{proof}
        We define the predicate 
        \[
        P(w) \defeq \delta^*(q_0, w) = 
        \begin{cases} 
            q_0 & \text{$\iff \abs w < 1$} \\
            q_1 & \text{$\iff w$ starts with a zero and ends with a zero} \\
            q_2 & \text{$\iff w$ starts with a one and ends with a one} \\
            q_3 & \text{$\iff w$ starts with a zero and ends with a one} \\
            q_4 & \text{$\iff w$ starts with a one and ends with a zero}
        \end{cases}
        \]
        These are our five state invariants that we want to prove.
        
        \newpage
        
        \base
        
        Show that the empty string $\epsilon$ satisfies the state invariant of 
        the initial state: $w = \epsilon$.
        
        In our case the initial state is $q_0$ and since the
        length of the empty string is zero, this invariant holds. Thus
        $P(\epsilon)$ is true.
        
        % In our case the initial state is $q_0$ and since the
        % empty string neither contains a first of last character, this 
        % invariant is vacuously true.
        
        \istep 
        
        For each transition $q \xrightarrow{\alpha}r$, show that if a string $w$ satisfies the
        invariant of state $q$, then the string $w \alpha$ satisfies the invariant 
        of $r$: 
        
        There are ten transition states we must show that this is true.
        Note that the structure of our DFA is symmetric with respect to the
        transition states, so we will show that the invariants of the top
        half hold without loss of generality. The bottom half is shown 
        similarly with the $0$'s and $1$'s switched (which we show in brackets).
        % Note that the structure of our DFA is symmetric with respect to the 
        % transition states, so we will prove that the transition states of 
        % the top half of the DFA satisfy their invariants with the  
        % understanding that the bottom half's state invariants are shown in 
        % the exact same manner except reversing the $0$'s and $1$'s. We leave it to
        % the reader to show the bottom half. Jk!


%                           ===> METHOD 1 <===        
        % Let $w = \gamma \alpha$, where $\gamma \in \{ 0, 1 \}^*$ and 
        % $\alpha \in \{ 0, 1 \}$. Assume $P(\gamma)$ holds. We want to that show
        % $P(w)$ is true for each transition state.
        
        % \vspace{5mm}
        
        % Consider two cases: $\alpha = 0, 1$. Let $\alpha = 0$. Then we have
        % \begin{align*}
        %     \delta^*(q_0, w) &= \delta(\delta^*(q_0, \gamma), 0) \\
        %     &=  \begin{cases}
        %             \delta(q_0, 0) = q_1 & \text{$\iff \gamma = \epsilon$} \\
        %             \delta(q_1, 0) = q_1 & \text{$\iff \gamma$ starts with a 
        %             zero and ends with a zero} \\
        %             \delta(q_2, 0) = q_4 & \text{$\iff \gamma$ starts with a 
        %             one and ends with a one} \\
        %             \delta(q_3, 0) = q_1 & \text{$\iff \gamma$ starts with a 
        %             zero and ends with a one} \\
        %             \delta(q_4, 0) = q_4 & \text{$\iff \gamma$ starts with a 
        %             one and ends with a zero}
        %         \end{cases}
        % \end{align*}
        
        % \vspace{5mm}
        
        % On the other hand, let $\alpha = 1$. Then we have
        % \begin{align*}
        %     \delta^*(q_0, w) &= \delta(\delta^*(q_0, \gamma), 1) \\
        %     &=  \begin{cases}
        %             \delta(q_0, 1) = q_2 & \text{$\iff \gamma = \epsilon$} \\
        %             \delta(q_1, 1) = q_3 & \text{$\iff \gamma$ starts with a 
        %             zero and ends with a zero} \\
        %             \delta(q_2, 1) = q_2 & \text{$\iff \gamma$ starts with a 
        %             one and ends with a one} \\
        %             \delta(q_3, 1) = q_3 & \text{$\iff \gamma$ starts with a 
        %             zero and ends with a one} \\
        %             \delta(q_4, 1) = q_2 & \text{$\iff \gamma$ starts with a 
        %             one and ends with a zero}
        %         \end{cases}
        % \end{align*}
        
        % Thus $P(w)$ holds for each transition state, and this completes the
        % induction step. Having shown the base case and induction step, we 
        % can conclude that the DFA $M$ is true for all strings $w$ over 
        % $\{ 0, 1 \}^*$. Therefore, $\cL(R) = \cL(M) = L_1$.
%                           ===> METHOD 1 END <===
        
%                           ===> METHOD 2 START <===
        \vspace{5mm}
        
        \textbf{1. Transition states $q_0 \xrightarrow{0} q_1$ and $q_0 \xrightarrow{1} q_2$:}
        
        Suppose that an arbitrary 
        string $w$ satisfies the invariant of state $q_0$ (i.e. it's just 
        $w = \epsilon$ since $q_0$ is the initial state). Appending a $0$ ($1$)
        from the string gives us $w0 = \epsilon 0 = 0$ ($w1 = \epsilon 1 = 1$).
        Thus $w0$ ($w1$) starts and ends with the same character since it 
        has been the only character read so far. Hence $w0$ ($w1$) satisfies
        the invariant of $q_1$ ($q_2$). Thus
        \begin{align*}
            \delta(q_0, w0) &= \delta(\delta(q_0, w), 0) \\
            &= \delta(q_0, 0) \tag{by assumption} \\
            &= q_1
        \end{align*}
        ($\delta(q_0, w1) = \delta(\delta(q_0, w), 1) = q_2$).
        
        \vspace{5mm}
        
        \textbf{2. Transition states $q_1 \xrightarrow{0} q_1$ and $q_2 \xrightarrow{1} q_2$:}
        
        Assume an arbitrary string $w$ satisfies the invariant of state
        $q_1$ ($q_2$). Appending a $0$
        ($1$) to the string doesn't change the first character, and since 
        the first and last character was already a zero (one) by assumption, 
        appending a zero (one) doesn't change the last character.
        Hence $w0$ ($w1$) satisfies the invariant of $q_1$ ($q_2$). 
        Thus 
        \begin{align*}
            \delta(q_0, w0) &= \delta(\delta(q_0, w), 0) \\
            &= \delta(q_1, 0) \tag{by assumption} \\
            &= q_1
        \end{align*}
        ($\delta(q_0, w1) = \delta(\delta(q_0, w), 1) = q_2$).
        
        \newpage
        
        \textbf{3. Transition states $q_1 \xrightarrow{1} q_3$ and $q_2 \xrightarrow{0} q_4$:}
        
        Assume an arbitrary string $w$ satisfies state $q_1$ ($q_2$). Appending a
        $1$ ($0$) doesn't change the first character, but does change the last 
        character to a $1$ ($0$). Hence $w1$ ($w0$) satisfies the invariant of 
        $q_3$ ($q_4$). That is,
        \begin{align*}
            \delta(q_0, w1) &= \delta(\delta(q_0, w), 1) \\
            &= \delta(q_1, 1) \tag{by assumption} \\
            &= q_3
        \end{align*}
        ($\delta(q_0, w0) = \delta(\delta(q_0, w), 0) = q_4$).
        
        \vspace{5mm}
        
        \textbf{4. Transition states $q_3 \xrightarrow{0} q_1$ and $q_4 \xrightarrow{1} q_2$:}
        
        Suppose an arbitrary string $w$ satisfies state $q_3$ ($q_4$). Appending a 
        $0$ ($1$) to $w$ only changes the last character to a $0$ ($1$). Hence
        $w0$ ($w1$) satisfies the invariant of $q_1$ ($q_2$). That is,
        \begin{align*}
            \delta(q_0, w0) &= \delta(\delta(q_0, w), 0) \\
            &= \delta(q_3, 0) \tag{by assumption} \\
            &= q_1
        \end{align*}
        ($\delta(q_0, w1) = \delta(\delta(q_0, w), 1) = q_2$).
        
        \vspace{5mm}
        
        \textbf{5. Transition states $q_3 \xrightarrow{1} q_3$ and $q_4 \xrightarrow{0} q_4$:}
        
        Suppose an arbitrary string $w$ satisfies state $q_3$ ($q_4$). Appending 
        a $1$ ($0$) to $w$ only changes the last character to a $1$ ($0$). Thus
        $w1$ ($w0$) satisfies the invariant of $q_3$ ($q_4$). Hence
        \begin{align*}
            \delta(q_0, w1) &= \delta(\delta(q_0, w), 1) \\
            &= \delta(q_3, 1) \tag{by assumption} \\
            &= q_3
        \end{align*}
        ($\delta(q_0, w0) = \delta(\delta(q_0, w), 0) = q_4$).
        
        \vspace{5mm}
        
        Thus $P(w)$ holds for each transition state, and this completes the
        induction step. Having shown the base case and induction step, we 
        can conclude that the DFA $M$ is true for all strings $w$ over 
        $\{ 0, 1 \}^*$. Therefore, $\cL(R) = \cL(M) = L_1$.
    \end{proof}
	
	
\newpage
	
	\subquestion{8}
	Let a \emph{block} be a maximal sequence of identical characters in a finite string. 
	For example, the string $0010101111$ can be broken up into blocks: 
	$00$, $1$, $0$, $1$, $0$, $1111$. Let $L_2 = \{x \in \{0,1\}^*: x \mbox{ only 
	contains blocks of length at least three}\}$.
	
	We define the regular expression 
	$R = (0^3 0^* + 1^3 1^*)^*$ 
	such that
	$\cL(R) = L_2$. That is, $L_2$ is the regular language matched by the regular 
	expression $R$.
	
	Next, we define the DFA $M$:
	
	\vspace{5mm}
	
	\begin{tikzpicture}[shorten >=1pt,node distance=4cm,on grid,auto]
	    \node[state,initial]    (q_0)                           {$q_0$};
	    \node[state]            (q_1)   [above right=of q_0]    {$q_1$};
	    \node[state]            (q_2)   [right=of q_1]          {$q_2$};
	    \node[state,accepting]  (q_3)   [right=of q_2]          {$q_3$};
	    \node[state]            (q_4)   [below right=of q_0]    {$q_4$};
	    \node[state]            (q_5)   [right=of q_4]          {$q_5$};
	    \node[state,accepting]  (q_6)   [right=of q_5]          {$q_6$};
	    \node[state]            (q_d)   [below left=of q_2]     {$q_d$};
	    
	    \path[->]
	    (q_0)   edge    [bend left]     node              {0}   (q_1)
	            edge    [bend right]    node    [swap]    {1}   (q_4)
	    (q_1)   edge    [bend left]     node              {0}   (q_2)
	            edge                    node    [swap]    {1}   (q_d)
	    (q_2)   edge    [bend left]     node              {0}   (q_3)
	            edge                    node    [swap]    {1}   (q_d)
	    (q_3)   edge    [loop right]    node              {0}   ()
	            edge    [bend left]     node              {1}   (q_4)
	    (q_4)   edge    [bend right]    node    [swap]    {1}   (q_5)
	            edge                    node              {0}   (q_d)
	    (q_5)   edge    [bend right]    node    [swap]    {1}   (q_6)
	            edge                    node    [swap]    {0}   (q_d)
	    (q_6)   edge    [loop right]    node              {1}   ()
	            edge    [bend right]    node    [swap]    {0}   (q_1)
	    (q_d)   edge    [loop right]    node              {0, 1}    ();
	\end{tikzpicture}
	
	\textbf{Proof of Correctness}
    \begin{proof}
        We define the predicate 
        \[
        P(w) \defeq \delta^*(q_0, w) = 
        \begin{cases} 
            q_0 & \text{$\iff \abs w < 1$} \\
            q_1 & \text{$\iff w$ ends with a block of $0^1$} \\
            q_2 & \text{$\iff w$ ends with a block of $0^2$} \\
            q_3 & \text{$\iff w$ only contains blocks of length at least three and
            ends with a block of $0^n$} \\
            q_4 & \text{$\iff w$ ends with a block of $1^1$} \\
            q_5 & \text{$\iff w$ ends with a block of $1^2$} \\
            q_6 & \text{$\iff w$ only contains blocks of length at least three and
            ends with a block of $1^n$} \\
            q_d & \text{$\iff w$ contains at least one segment block 
            \footnotemark}
        \end{cases}
        \]
        \footnotetext{We define
        a segment block as a sequence such as $0^j 1^k$ or $1^j 0^k$, where $0 < j < 3$
        and $k \in \Z^+$.}
        for $n \geqslant 3 \in \N$.
        These are our eight state invariants that we want to prove.
        
        \newpage
        
        \base
        
        Show that the empty string $\epsilon$ satisfies the state invariant of 
        the initial state: $w = \epsilon$.
        
        In our case the initial state is $q_0$ and since the
        length of the empty string is zero, this invariant holds. Thus
        $P(\epsilon)$ is true.
        
        \istep 
        
        For each transition $q \xrightarrow{\alpha}r$, show that if a string $w$ satisfies the
        invariant of state $q$, then the string $w \alpha$ satisfies the invariant 
        of $r$: 
        
        There are fifteen transition states to consider.
        Note that, once again, the structure of our DFA is 
        symmetric with respect to the
        transition states, so we will show that the invariants of the top
        half hold without loss of generality. The bottom half is shown 
        similarly with the $0$'s and $1$'s switched (which we show in brackets).
        
        \vspace{5mm}
        
        \textbf{1. Transition states $q_0 \xrightarrow{0} q_1$ and $q_0 \xrightarrow{1} q_4$:}
        
        Suppose that an arbitrary 
        string $w$ satisfies the invariant of state $q_0$ (i.e. it's just 
        $w = \epsilon$ since $q_0$ is the initial state). Appending a $0$ ($1$)
        from the string gives us $w0 = \epsilon 0 = 0$ ($w1 = \epsilon 1 = 1$).
        Thus $w0$ ($w1$) ends with a block of $0^1$ ($1^1$). Hence $w0$ ($w1$) satisfies
        the invariant of $q_1$ ($q_4$). Thus
        \begin{align*}
            \delta(q_0, w0) &= \delta(\delta(q_0, w), 0) \\
            &= \delta(q_0, 0) \tag{by assumption} \\
            &= q_1
        \end{align*}
        ($\delta(q_0, w1) = \delta(\delta(q_0, w), 1) = q_4$).
        
        \vspace{5mm}
        
        \textbf{2. Transition states $q_1 \xrightarrow{0} q_2$ and $q_4 \xrightarrow{1} q_5$:}
        
        Assume an arbitrary string $w$ satisfies the invariant of state $q_1$ ($q_4$). 
        Appending a $0$ ($1$) to $w$ adds another zero (one) to the end of $w$. Hence
        $w0$ ($w1$) ends with a block of $0^2$ ($1^2$). That is,
        \begin{align*}
            \delta(q_0, w0) &= \delta(\delta(q_0, w), 0) \\
            &= \delta(q_1, 0) \tag{by assumption} \\
            &= q_2
        \end{align*}
        ($\delta(q_0, w1) = \delta(\delta(q_0, w), 1) = q_5$).
        
        \vspace{5mm}
        
        \textbf{3. Transition states $q_2 \xrightarrow{0} q_3$ and $q_5 \xrightarrow{1} q_6$:}
        
        Assume an arbitrary string $w$ satisfies the invariant of state $q_2$ ($q_5$). 
        Appending a $0$ ($1$) to $w$ adds another zero (one) to the end of $w$. Hence
        $w0$ ($w1$) ends with a block of $0^3$ ($1^3$). That is,
        \begin{align*}
            \delta(q_0, w0) &= \delta(\delta(q_0, w), 0) \\
            &= \delta(q_2, 0) \tag{by assumption} \\
            &= q_3
        \end{align*}
        ($\delta(q_0, w1) = \delta(\delta(q_0, w), 1) = q_6$).
        
\newpage
        
        \textbf{4. Transition states $q_3 \xrightarrow{0} q_3$ and $q_6 \xrightarrow{1} q_6$:}
        
        Assume an arbitrary string $w$ satisfies the invariant of state $q_3$ ($q_6$). 
        Appending a $0$ ($1$) to $w$ adds another zero (one) to the end of $w$. Hence
        $w0$ ($w1$) ends with a block of at least three. That is, 
        $w0$ ($w1$) ends with a block of $0^n$ ($1^n$), where $n \geqslant 3 \in \N$. Hence,
        \begin{align*}
            \delta(q_0, w0) &= \delta(\delta(q_0, w), 0) \\
            &= \delta(q_3, 0) \tag{by assumption} \\
            &= q_3
        \end{align*}
        ($\delta(q_0, w1) = \delta(\delta(q_0, w), 1) = q_6$).
        
        \vspace{5mm}
        
        \textbf{5. Transition states $q_3 \xrightarrow{1} q_4$ and $q_6 \xrightarrow{0} q_1$:}
        
        Assume an arbitrary string $w$ satisfies the invariant of state $q_3$ ($q_6$). 
        Appending a $1$ ($0$) to $w$ adds a one (zero) to the end of $w$. Hence
        $w1$ ($w0$) ends with a block of $1^1$ ($0^1$). Hence,
        \begin{align*}
            \delta(q_0, w1) &= \delta(\delta(q_0, w), 1) \\
            &= \delta(q_3, 1) \tag{by assumption} \\
            &= q_4
        \end{align*}
        ($\delta(q_0, w0) = \delta(\delta(q_0, w), 0) = q_1$).
        
        \vspace{5mm}
        
        \textbf{6. Transition states $q_4 \xrightarrow{0} q_d$ and $q_1 \xrightarrow{1} q_d$:}
        
        Assume an arbitrary string $w$ satisfies the invariant of state $q_4$ ($q_1$). 
        Appending a $0$ ($1$) to $w$ adds another zero (one) to the end of $w$. Since
        $w$ ends with a block of $1^1$ ($0^1$) by our assumption, then
        $w0$ ($w1$) ends with a block of $0^1$ ($1^1$). But by our assumption of $w$, 
        we also
        know that the block before appending a $0$ ($1$) to the end of $w$ only 
        contained a block of $1^1$ ($0^1$). Then since $w0$ ($w1$) contains at least
        one block segment (as define above $j < 3$), so we can conclude that 
        $w0$ ($w1$) satisfies the state invariant of $q_d$. That is,
        \begin{align*}
            \delta(q_0, w0) &= \delta(\delta(q_0, w), 0) \\
            &= \delta(q_4, 0) \tag{by assumption} \\
            &= q_d
        \end{align*}
        ($\delta(q_0, w1) = \delta(\delta(q_0, w), 1) = q_d$).
        
\newpage
        
        \textbf{7. Transition states $q_5 \xrightarrow{0} q_d$ and $q_2 \xrightarrow{1} q_d$:}
        
        Assume an arbitrary string $w$ satisfies the invariant of state $q_5$ ($q_2$). 
        Appending a $0$ ($1$) to $w$ adds a zero (one) to the end of $w$. Then
        $w0$ ($w1$) ends with a block of $0^1$ ($1^1$). But by our assumption of $w$, 
        we also
        know that the block before appending a $0$ ($1$) to the end of $w$ only 
        contained a block of $1^2$ ($0^2$). Then since $w0$ ($w1$) contains at least
        one block segment (as define above $j < 3$), so we can conclude that 
        $w0$ ($w1$) satisfies the state invariant of $q_d$. That is,
        \begin{align*}
            \delta(q_0, w0) &= \delta(\delta(q_0, w), 0) \\
            &= \delta(q_5, 0) \tag{by assumption} \\
            &= q_d
        \end{align*}
        ($\delta(q_0, w1) = \delta(\delta(q_0, w), 1) = q_d$).
        
        \vspace{5mm}
        
        \textbf{8. Transition state $q_d \xrightarrow{0, 1} q_d$:}
        
        Assume an arbitrary string $w$ satisfies the invariant of state $q_d$. 
        Appending a $0$ or $1$ to $w$ adds a zero or one to the end of $w$. But
        by our assumption of $w$, we know that $w$ contains at least one segment block
        such that $0^j1^k$ or $1^j0^k$ and $0 < j < 3$. So appending a $0$ or $1$
        to $w$ doesn't change what we already know about $w$, 
        (i.e. we still know that $w$ contains at least one segment block)
        Thus we can conclude that $w0$ and $w1$ satisfy the state invariant 
        of $q_d$. That is,
        \begin{align*}
            \delta(q_0, w0) &= \delta(\delta(q_0, w), 0) \\
            &= \delta(q_d, 0) \tag{by assumption} \\
            &= q_d
        \end{align*}
        
        Also,
        \begin{align*}
            \delta(q_0, w1) &= \delta(\delta(q_0, w), 1) \\
            &= \delta(q_d, 1) \tag{by assumption} \\
            &= q_d
        \end{align*}
        
        Thus $P(w)$ holds for each transition state, and this completes the
        induction step. Having shown the base case and induction step, we 
        can conclude that the DFA $M$ is true for all strings $w$ over 
        $\{ 0, 1 \}^*$. Therefore, $\cL(R) = \cL(M) = L_2$.
    \end{proof}
	
	
	
	
	
%               ===> DFA example <===
% \begin{tikzpicture}[shorten >=1pt,node distance=2cm,on grid,auto] 
%   \node[state,initial] (q_0)   {$q_0$}; 
%   \node[state] (q_1) [above right=of q_0] {$q_1$}; 
%   \node[state] (q_2) [below right=of q_0] {$q_2$}; 
%   \node[state,accepting](q_3) [below right=of q_1] {$q_3$};
%     \path[->] 
%     (q_0) edge  node {0} (q_1)
%           edge  node [swap] {1} (q_2)
%     (q_1) edge  node  {1} (q_3)
%           edge [loop above] node {0} ()
%     (q_2) edge  node [swap] {0} (q_3) 
%           edge [loop below] node {1} ();
% \end{tikzpicture}

%               ===> DFA example 2<===
% \begin{tikzpicture}[shorten >=1pt,node distance=2cm,on grid]
%   \node[state,initial]   (q_0)                {$q_0$};
%   \node[state]           (q_1) [right=of q_0] {$q_1$};
%   \node[state,accepting] (q_2) [right=of q_1] {$q_2$};
%   \path[->] (q_0) edge                node [above] {0} (q_1)
%                   edge [loop above]   node         {1} ()
%                   edge [bend left=45] node [below] {1} (q_2)
%                   edge [bend right]   node [below] {0} (q_2)
%             (q_1) edge                node [above] {1} (q_2);
% \end{tikzpicture}