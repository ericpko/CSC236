\question{16}

Given a list $L$, a \textit{contiguous}
sublist $M$ of $L$ is a sublist of $L$
whose elements occur in immediate succession in $L$.
For instance, $[4,7,2]$ is a contiguous sublist of $[0,4,7,2,4]$
but $[4,7,2]$ is not a contiguous sublist of $[0,4,7,1,2,4]$.

We consider the problem of computing,
for a list of integers $L$,
a contiguous sublist $M$ of $L$
with maximum possible sum.

\begin{minipage}{\linewidth}
    \begin{algorithm}[H]
        \caption{$MaxSublist(L)$}
        \begin{algorithmic}[1]

            \vspace{.5 em}
            \Require $L$ is a list of integers.
            \Ensure Return the maximum sum of a contiguous sublist of $L$.
            \vspace{.5 em}

        \end{algorithmic}
    \end{algorithm}
\end{minipage}

\subquestion{5}
    Using a divide-and-conquer approach, devise a recursive algorithm 
    which meets the requirements of $MaxSublist$.
    
    \begin{minted}[xleftmargin=20pt,linenos]{python}
    def max_sublist_sum(lst: List[int]) -> int:
        """
        Pre: <lst> is a list of integers.
        Post: returns a contiguous sublist of <lst> with the 
        maximum possible sum.
        Note: This could also be implemented with indices.
        """
        n = len(lst)

        # base case
        if n <= 1:
            return max(lst[0], 0)  # empty sublist is zero, 
                                   # which is better than negative

        else:
            # divide the list
            mid = n // 2        # note: n is at least 2, so mid is at least 1
    
            # conqueur the list
            left_sum = max_sublist_sum(lst[:mid])   # from index 0...mid - 1
            right_sum = max_sublist_sum(lst[mid:])  # from index mid...n - 1
            cross_sum = max_crossing_sum(lst, mid, n)
    
            return max(left_sum, right_sum, cross_sum)
    \end{minted}
    
\newpage
    
        \begin{minted}[xleftmargin=20pt,linenos]{python}
    def max_crossing_sum(lst: List[int], mid: int, n: int) -> int:
        """
        Parameter <mid> is the floor middle index of <lst>.
        Parameter <n> is the length of the input list <lst>.
        Pre: <lst> is a list of integers and len(lst) >= 2.
        Post: returns the maximum contiguous crossing sum 
        starting from the middle of <lst>.
        """
        left_sum, right_sum, total = 0, 0, 0 # initialize values
    
        # max sum the left half
        k = mid - 1
        i = 0
        while i < mid:
            total += lst[k - i]
            i += 1
            if total > left_sum:
                left_sum = total
    
        i, total = 0, 0             # reset values
        # max sum the right half
        for i in range(mid, n):     # iterate from index mid...n - 1
            total += lst[i]
            if total > right_sum:
                right_sum = total
    
        # note: left_sum and right_sum are each at least zero
        crossing_sum = left_sum + right_sum
        return crossing_sum
    \end{minted}

\newpage

\subquestion{8}
    Give a complete proof of correctness for your algorithm.
    If you use an iterative subprocess, prove the correctness of this also.
    
    We will first prove the correctness of $max\_crossing\_sum$.
    \begin{proof}
    \textbf{Loop 1}
    
    For this proof, let $A = \texttt{lst}$ and $m = \texttt{len(A[:mid])}$. 
    Define the loop invariant: 
    $$Inv(i, total, left\_sum): 0 \leqslant i \leqslant m \AND 
    total = \sum_{j = 0}^{i - 1}A[k - j] 
    \AND left\_sum_i = \max{(total_i, left\_sum_{i-1})}$$
    where we define $left\_sum_{-1}$ to be zero.
    
    \base
    
    When the loop is reached, $i = 0$, 
    $total = 0$, and $left\_sum_0 = 0$. So our invariant is
    $$Inv(0, 0, 0): 0 \leqslant 0 \leqslant m 
    \AND 0 = \sum_{j = 0}^{- 1}A[k - j]
    \AND 0 = \max{(0, 0)}$$
    which is true.
    
    \istep
    
    Assume that the invariant is true before an arbitrary iteration
    of the loop. Let $i_0$, $total_0$, $left\_sum_0$, 
    $i_1$, $total_1$, $left\_sum_1$ be the values of 
    $i$, $total$, and $left\_sum$ before and after an arbitrary loop, 
    respectively.
    Assume
    $Inv(i_0, total_0, left\_sum_0)$ is true and that  $i_0 < m$.
    We want to show that $Inv(i_1, total_1, left\_sum_1)$ is true. That is,
    \[
    Inv(i_1, total_1, left\_sum_1):
    \]
    \[
    0 \leqslant i_1 \leqslant m 
    \AND total_1 = \sum_{j = 0}^{i_1 - 1}A[k - j]
    \AND left\_sum_1 = \max{(total_1, left\_sum_0)}
    \]
    
    From the loop body, we have
    \begin{itemize}
        \item [(a)] $i_1 = i_0 + 1 \implies i_0 = i_1 - 1$
        \item [(b)] $total_1 = total_0 + A[k - i_0]$
        \item [(c)] $left\_sum_1 = \max{(total_1, left\_sum_0)}$
    \end{itemize}
    
    \vspace{5mm}
    Showing conjunct 1: $0 \leqslant i_1 \leqslant m$
    
    The first part is true because 
    $0 \leqslant i_0 \implies 1 \leqslant i_0 + 1 = i_1$. The second part
    is true because
    $i_0 < m \implies i_0 + 1 = i_1 \leqslant m$. 
    
    Showing conjunct 2: $total_1 = \sum_{j = 0}^{i_1 - 1}A[k - j]$.
    From the loop body, we have
    \begin{align*}
        total_1 &= total_0 + A[k - i_0] \\
        &= \sum_{j = 0}^{i_0 - 1}A[k - j] + A[k - i_0] \\
        &= \sum_{j = 0}^{i_0}A[k - j] \\
        &= \sum_{j = 0}^{i_1 - 1}A[k - j] \tag{by (a)} \\
    \end{align*}
    
    Showing conjunct 3: $left\_sum_1 = \max{(total_1, left\_sum_0)}$.
    
    But this is exactly what we have in the body of our loop for 
    part $(c)$. Note that $left\_sum_i$ is at least zero since we defined
    $left\_sum_{-1} = 0$. That is, it's base case is defined as zero.
    Thus we can conclude that $left\_sum$ is the maximum contiguous 
    sum of integers starting from the middle of the given list and 
    iterating to the left.
    
    \vspace{10mm}
    
    \textbf{Loop 2}
    
    For the second loop, we define a new loop invariant. For this
    loop invariant, let $m = mid$ from the code:
    $$Inv(i, total, right\_sum): m \leqslant i \leqslant n 
    \AND total = \sum_{j = m}^{i - 1}A[j]
    \AND right\_sum_i = \max{(total_i, right\_sum_{i-1})}$$
    where again, we define $right\_sum_{-1} = 0$.
    
    \base
    
    When the second loop is reached, $i = m$, $total = 0$, 
    $right\_sum = 0$. Then we have
    \[
    Inv(m, 0, 0): m \leqslant m \leqslant n
    \AND 0 = \sum_{j = m}^{m - 1}A[j]
    \AND 0 = \max{(0, 0)}
    \]
    which is true.
    Note that the sum above is equal to zero by definition of a sum 
    over an empty range.
    
    \istep
    
    Assume that the invariant is true before an arbitrary iteration
    of the loop. Let $i_0$, $total_0$, $right\_sum_0$, 
    $i_1$, $total_1$, $right\_sum_1$ be the values of 
    $i$, $total$, and $right\_sum$ before and after an arbitrary loop, 
    respectively.
    Assume
    $Inv(i_0, total_0, right\_sum_0)$ is true. Also assume
    that $i_0 < n$ so another iteration is necessary.
    We want to show that $Inv(i_1, total_1, right\_sum_1)$ is true. That is,
    \[
    Inv(i_1, total_1, right\_sum_1):
    \]
    \[
    m \leqslant i_1 \leqslant n 
    \AND total_1 = \sum_{j = m}^{i_1 - 1}A[j]
    \AND right\_sum_1 = \max{(total_1, right\_sum_{0})}
    \]
    
    From the second loop's body we have
    \begin{enumerate}
        \item [(a)] $i_1 = i_0 + 1$
        \item [(b)] $total_1 = total_0 + A[i_0]$
        \item [(c)] $right\_sum = \max{(total_1, right\_sum_0)}$
    \end{enumerate}
    
    Showing conjunct 1: $m \leqslant i_1 \leqslant n$.
    
    The first part is true since $m \leqslant i_0 < i_1$.
    The second part is also true by our assumption that 
    $i_0 < n$. So it follows that $i_1 \leqslant n$ by $(a)$ above.
    
    \vspace{5mm}
    
    Showing conjunct 2: $total_1 = \sum_{j = m}^{i_1 - 1}A[j]$.
    
    From the loop body we have
    \begin{align*}
        total_1 &= total_0 + A[i_0] \\ 
        &= \sum_{j = m}^{i_0 - 1}A[j] + A[i_0] \tag{by I.H.} \\
        &= \sum_{j = m}^{i_0}A[j] \\
        &= \sum_{j = m}^{i_1 - 1}A[j] \tag{by (a)}
    \end{align*}
    
    Showing conjunct 3: $right\_sum = \max{(total_1, right\_sum_0)}$
    
    But this is exactly what we have in property $(c)$ from the loop body.
    
    \vspace{10mm}
    
    \textbf{Step 4: Checking the post-condition}
    \vspace{5mm}
    
    We will now check that the invariants and negation of their
    respective loop guards imply the post-condition. Loop $1$'s guard
    is $i < m$ and terminates when $i \geqslant m$. That is, the
    negation of the guard implies termination.
    
    From loop 1, the invariant and negation of loop guard together:
    \[
    Inv(i, total, left\_sum): 
    \]
    \[
    0 \leqslant i \leqslant m
    \AND total = \sum_{j = 0}^{i - 1}A[k - j] 
    \AND left\_sum_i = \max{(total_i, left\_sum_{i-1})}
    \AND i \geqslant m
    \]
    implies that $i = m$. Thus $total = \sum_{j = 0}^{m - 1}A[k - j]$
    is the sum of all the integers in list $A$ from index $0$ up to 
    index $m - 1$. Therefore, $left\_sum$ is the maximum value of 
    a contiguous sum of integers starting from the middle of the given 
    list and iterating in reverse.
    
    
    From loop 2, the invariant and negation of loop guard together:
    \[
    Inv(i, total, right\_sum): 
    \]
    \[
    m \leqslant i \leqslant n
    \AND total = \sum_{j = m}^{i - 1}A[j] 
    \AND right\_sum_i = \max{(total_i, right\_sum_{i-1})}
    \AND i \geqslant n
    \]
    implies that $i = n$. Thus $total = \sum_{j = m}^{n - 1}A[j]$, 
    the sum of the integers in the list $A$ from index $m = mid$
    up to index $n - 1$. Also, this implies
    that $right\_sum$ is the maximum value of a contiguous sum of
    integers starting from the middle of the given list and 
    iterating to the right.
    
    Since the program returns $left\_sum + right\_sum$, we can conclude
    that $max\_crossing\_sum$ returns the maximum contiguous crossing sum of
    integers starting from the middle and expanding outward in both
    directions. Therefore the post-condition is satisfied.
    
    \vspace{10mm}
    \textbf{Step 5: Termination}
    
    \vspace{5mm}
    We define the following loop variants for loops $1$ and $2$,
    respectively:
    
    For loop 1, let $var_1(i) = m - i$, and let 
    $var_2(i) = n - i$. Since $i$ increases by $1$ each iteration, 
    $var_1(i)$, and $var_2(i)$ will decrease by $1$ each iteration.
    Also, from the invariants we know that
    $i \leqslant m$ and $i \leqslant n$, so it follows that 
    $m - i \geqslant 0$ and $n - i \geqslant 0$. Since each loop
    variant decreases on each iteration, but cannot decrease below
    zero, we can conclude that at some point each loop must terminate.
    
    \end{proof}

\newpage

    \textbf{Proof of $\textbf{max\_sublist\_sum}$}
    \begin{proof}
    
    Let $L$ be the input list to $max\_sublist\_sum$ and assume
    $L$ satisfies the precondition.
    
    \base 
    \emph{pre $\implies$ post}
    
    If \texttt{len(L)} $\leqslant 1$, then there is only one item
    in the list and the max of that item and zero is returned. 
    Since a contiguous sublist of $L$ with maximum possible
    sum is either the value of the only item, or zero, the 
    postcondition is satisfied. Note that an empty sublist
    has a maximum sum of zero, which is greater than a 
    negative item.
    
    \vspace{5mm}
    
    \istep
    
    \emph{pre $\implies$ pre}
    
    Assume $L$ satisfies the precondition and that \texttt{len(L)}
    $> 1$. Let $m = \texttt{len(L) // 2}$, $L' = L[:m]$, and 
    $L'' = L[m:]$.
    Further, we see that $L'$ contains the first half
    of the items in $L$ and $L''$ contain the second half of $L$.
    Since it is exactly $L'$ and $L''$ that is passed into the two
    recursive calls, we can conclude that their preconditions
    are satisfied.
    
    \emph{post $\implies$ post}
    
    Let $S' = max\_sublist\_sum(L')$ and $S'' = max\_sublist\_sum(L'')$.
    Assume $S'$ and $S''$ both satisfy the post condition; that is,
    they are each a contiguous sublist of their respective inputs 
    with the maximum possible value. But since $L'$ and $L''$ are
    the first and second halves of list $L$, then it follows that
    $S'$ is a contiguous sublist with the maximum possible value
    of the first half of $L$. Likewise, $S''$ is a contiguous 
    sublist with the maximum possible value of the second half of
    list $L$. But if the contiguous sublist with the maximum 
    possible value of $L$ is across the center of both both halves
    of $L$? In that case, let $CS = max\_crossing\_sum(L)$. Since
    we have proven the correctness of $max\_crossing\_sum$, we
    know $CS$ is exactly the maximum value we are missing from
    the two recursive calls. Lastly, since the program returns
    the $\max{(S', S'', CS)}$ we can conclude that the 
    postcondition of this algorithm is satisfied because the 
    maximum is necessarily in one of those three cases.
    
    \vspace{5mm}
    
    \textbf{Termination}
    
    We define a measure $m(L) = \texttt{len(L)}$. 
    Further, we will assume that
    $len(L') = \ceil{\frac{len(L)}{2}}$.
    Using the same
    symbols as above, we can see that each recursive call is made
    on a smaller list. We will show this is for $L'$, but note that
    the exact same argument holds for $L''$ since $L''$ is at most
    half the size of $L$.
    Thus we have
    \begin{align*}
        m(L') &= len(L') \\
        &= \ceil{\frac{len(L)}{2}} \tag{by def of L'} \\
        &< len(L) \\
        &= m(L)
    \end{align*}
    Thus each recursive call is made on a smaller input list until
    eventually a base case is reached.
    
    \end{proof}

\newpage

\subquestion{3}
    Analyze the running time of your algorithm.
    
    Let $n = len(lst)$ be the length of the input list to
    $max\_sublist\_sum$
    and let $T(n)$ be the runtime. If $n = 1$, then $T(1) = c$ for some 
    constant $c$. Otherwise there are two recursive calls made, each
    on a list of length $\frac{n}{2}$. Also, $max\_crossing\_sum$
    takes linear time since it iterates across the entire length of
    it's input list. So we end up with
    
    \[ 
    T(n) = 
    \begin{cases} 
        c & \text{if $n \leqslant 1$} \\
        2T(\frac{n}{2}) + dn + e & \text{otherwise}
    \end{cases}
    \]
    where $2T(\frac{n}{2})$ is the cost of the two recursive calls and
    $dn$ is the linear cost of $max\_crossing\_sum$, and finally
    $e$ is a constant for all the constants (i.e. \texttt{return}, etc.).
    Note that $2T(\frac{n}{2}) = T(\ceil{\frac{n}{2}}) + T(\floor{\frac{n}{2}})$.
    Then by the Master Theorem, $a = 2$, $b = 2$, $f(n) = dn + e$.
    Since $k = 1$ and $\log_{2}{2} = 1$, so the Master Theorem gives
    $$T(n) \in \mathcal{O}(n \log{n})$$

\newpage