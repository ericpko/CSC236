%!TEX root = csc236-a3.tex
\def \cL{\mathcal{L}}

\question{10}
The first three parts of this question deals with 
properties of regular expressions (this is question 4 from 
section 7.7 of Vassos' textbook). Two regular expressions 
$R$ and $S$ are equivalent, written $R \equiv S$ if their 
underlying language is the same i.e. $\mathcal{R} = \mathcal{S}$. 
Let $R, S$, and $T$ be arbitrary regular expression. For 
each assertion, state whether it is true or false and 
justify your answer. 

	\subquestion{2}
	\begin{center}
		If $RS \equiv SR$ then $R \equiv S$.	
	\end{center}
	
	\begin{proof}
	    Suppose this statement is true. Then lets define regular
	    expressions for $R$ and $S$ over the alphabet $\{ 0, 1 \}$.
	    Let $R = (0 + 1)$ and let $S = (0 + 1)^*$. Then at least 
	    intuitively, we can see that $RS \equiv SR$ holds.
	    But by our assignment of $R$ and $S$, we know that
	    $R \not\equiv S$. Thus we have reached a contradiction, so
	    our initial assumption was incorrect and we can conclude
	    that this statement is false.
	\end{proof}

	\subquestion{2}
	\begin{center}
		If $RS \equiv RT$ and $R \not\equiv \emptyset$ then $S \equiv T$.
	\end{center}
	
	\begin{proof}
	    Let $R, S, T$ be arbitrary regular expressions and assume
	    that 
	    \begin{itemize}
	        \item [(\emph{i})] $RS \equiv RT$ and
	        \item [(\emph{ii})] $R \not\equiv \emptyset$.
	    \end{itemize}
	    By the first assumption, if we take the regular languages
	    we get 
	    \begin{align}
	        \mathcal{L}(RS) &\equiv \mathcal{L}(RT) \\
	        \mathcal{L}(R) \mathcal{L}(S) &\equiv \mathcal{L}(R) 
	        \mathcal{L}(T)
	    \end{align}
	    Then equating both sides of ($2$) and using the fact that 
	    $\cL(R) \not\equiv \emptyset$ we get
	    \begin{itemize}
	        \item [(a)] $\mathcal{L}(R) \equiv \mathcal{L}(R)$ and 
	        \item [(b)] $\mathcal{L}(S) \equiv \mathcal{L}(T)$
	    \end{itemize}
	    Property (b) says that the two regular languages (sets)
	    represented by the regular expressions $S$ and $T$ are 
	    equivalent. Now we can use the property stated in the 
	    question such that if the underlying regular language 
	    is the same (i.e. equivalent), then it follows that 
	    the regular expressions are equivalent. That is, since
	    $\mathcal{L}(S) \equiv \mathcal{L}(T)$, then $S \equiv T$.
	    Therefore, we conclude that this statement is true.
	\end{proof}
	
\newpage

	\subquestion{2}
	\[
	(RS + R)^*R \equiv R(SR+R)^*.
	\]
	
	Let $R, S$ be arbitrary regular expressions.
	Take any string from the LHS and note that it always 
	begins with a character from the regular expression
	$R$ and ends with a character from $R$. In between
	these characters, there is any number of characters from
	$RS$ or $R$.
	
	On the other hand, any string from the RHS also
	always starts with and ends with a character from
	$R$. Also, in between these characters, there is 
	any number of characters from $SR$ or $R$. As we 
	have shown in \textbf{Part (1)}, it does not need
	to be the case that $R \equiv S$. Thus we can 
	conclude that this statement is true.
	
% 	Taking the regular language of both sides gives
	
% 	\begin{align}
% 	    \cL((RS + R)^*R) &\equiv \cL(R (SR + R)^*) \\
% 	    (\cL(RS) + \cL(R))^* \cL(R) &\equiv \cL(R) (\cL(SR) + \cL(R))^* \\
% 	    (\cL(R) \cL(S) + \cL(R))^* \cL(R) &\equiv \cL(R)
% 	    (\cL(S) \cL(R) + \cL(R))^*
% 	\end{align}
	
% 	To show that these two regular languages
% 	are equivalent, it is necessary to show that the LHS 
% 	$\subseteq$ RHS and the RHS $\subseteq$ LHS.

\newpage